\documentclass{article}

\usepackage{tabularx}
\usepackage{booktabs}
\usepackage{hyperref}

\setlength{\parskip}{1em}

\title{CAS 741: Problem Statement - \\PID Controller}

\author{Naveen Ganesh Muralidharan . muralidn}

\date{24-September-2020}

\input{../Comments}

\begin{document}

\maketitle

\begin{table}[hp]
\caption{Revision History} \label{TblRevisionHistory}
\begin{tabularx}{\textwidth}{>{\hsize=.1\hsize\linewidth=\hsize}X
>{\hsize=.2\hsize\linewidth=\hsize}X
>{\hsize=.6\hsize\linewidth=\hsize}X}
\toprule
\textbf{Date} & \textbf{Developer(s)} & \textbf{Change}\\
\midrule
21-Sep-2020 & Naveen Ganesh Muralidharan & First revision of the PID problem 
statement\\
24-Sep-2020 & Naveen Ganesh Muralidharan & Updates as per the latest problem
statement checklist, viz,
\begin{itemize}
\item Included the Comments.tex file. 
\item Wrapped text in the .tex file to 80 chars. 
\item Corrected grammatical mistakes.
\item Added reference to Drasil.
\end{itemize}\\
\bottomrule
\end{tabularx}
\end{table}

\section*{Problem Statement}

A closed-loop control system can be defined as the system where the input to the 
Power Plant is continuously adjusted by monitoring the feedback from the Power 
Plant until the expected Set-Point is reached. The closed-loop control is used 
in a variety of applications such as cruise control of an automobile, 
temperature control in a thermostat, and many more. The heart of the control 
loop is a Proportional, Integral, Derivative (PID) controller that drives the 
input to the Power Plant in the loop. However, the PID controller in a loop must 
be tuned before it is deemed ready for use. This involves setting optimal values 
for the respective Proportional, Integral, and Derivative gain constants. 
Therefore, a model is necessary to simulate a control loop, using which the 
PID gains can be tuned.

The inputs to the Model are the Set-Point (numeric), Proportional Gain 
(numeric), Integral Gain (numeric), Derivative Gain (numeric), Total 
Simulation Time (numeric), and Step-Time (numeric). The outputs from the Model 
are the Measured Values from the Power Plant (numeric list) for each iteration, 
and their respective Time-Points (numeric list).

This Model shall be implemented using the Drasil project. Following is the 
homepage of the Drasil project,

\url{https://jacquescarette.github.io/Drasil/} 

\section*{Stakeholders}
Primary stakeholders of this Project are,
\begin{itemize}
\item Naveen Ganesh Muralidharan
\item Dr. Spencer Smith
\item Students of the class CAS 741
\item Dr. Jacques Carette
\item All contributors of the Drasil Project
\end{itemize}

\section*{Software Environment}
The software application in this project is designed to execute on Ubuntu 18.04 
and Derivatives, Windows 10 and, macOS 10.13 and greater.

%\wss{comment}

%You can also leave comments for yourself, like this:

%\an{comment}

\end{document}
