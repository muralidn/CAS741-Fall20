\documentclass[12pt, titlepage]{article}

\usepackage{booktabs}
\usepackage{tabularx}
\usepackage{hyperref}


\usepackage{amsmath, mathtools}
\usepackage{amsfonts}
\usepackage{amssymb}
\usepackage{graphicx}
\usepackage{colortbl}
\usepackage{xr}
\usepackage{hyperref}
\usepackage{longtable}
\usepackage{xfrac}
\usepackage{float}
\usepackage{siunitx}
\usepackage{pdflscape}
\usepackage{afterpage}

\usepackage{longtable}
\usepackage{caption}
\usepackage{enumitem}

\hypersetup{
    colorlinks,
    citecolor=blue,
    filecolor=black,
    linkcolor=red,
    urlcolor=blue
}
\usepackage[square,numbers]{natbib}

\input{../Comments}
%% Common Parts

\newcommand{\progname}{PID Controller} % PUT YOUR PROGRAM NAME HERE %Every program
                                % should have a name


\begin{document}

\title{System Verification and Validation Plan for \progname{}} 
\author{Naveen Ganesh Muralidharan}
\date{\today}
	
\maketitle

\pagenumbering{roman}

\section{Revision History}

\begin{tabularx}{\textwidth}{p{3cm}p{2cm}X}
\toprule {\bf Date} & {\bf Version} & {\bf Notes}\\
\midrule
28-Oct-20 1 & 1.0 & The first draft of the VnV plan\\
\bottomrule
\end{tabularx}

\newpage

\tableofcontents

\listoftables

\newpage

\section{Symbols, Abbreviations, and Acronyms}

\renewcommand{\arraystretch}{1.2}
\begin{tabular}{l l} 
  \toprule		
  \textbf{symbol} & \textbf{description}\\
  \midrule 
  API & Application Program Interface\\
  N & No\\
  PEP8 & Python Enhancement Proposal 8\\
  stdin & Standard input stream\\
  stderr & Standard error stream\\
  stdout & Standard output stream\\
  T & Test\\
  VnV & Verification and Validation\\
  Y & Yes\\
  \bottomrule
\end{tabular}\\

\wss{symbols, abbreviations or acronyms -- you can simply reference the SRS
  \citep{SRS} tables, if appropriate}

All the units, abbreviations, and symbols recorded in the Software Requirement 
Specification \cite{SRS} apply to this document as well.

\newpage

\pagenumbering{arabic}

This document encompasses the Verification and Validation (VnV) plan for the
\progname{} software. 

Section 3 of this document sets the context for the VnV plan. Section 4  provides
a high-level overview of the VnV plan. Sections 5 and 6 contain the systems and 
unit test cases respectively. 

\wss{provide an introductory blurb and roadmap of the
  Verification and Validation plan}

\section{General Information}

\subsection{Summary}

\wss{Say what software is being tested.  Give its name and a brief overview of
  its general functions.}

The software under test is the simulation of a PID control loop. The functions of 
the PID control loop are,

\begin{itemize}
\item Calculating the Error Signal. Error Signal is the difference between the User 
Input (Set-Point) and the output of the Power Plant (Process-Variable).
\item Computing the output of the PID Controller.
\item Computing the response of the Power-Plant.
\end{itemize}

\subsection{Objectives}

\wss{State what is intended to be accomplished.  The objective will be around
  the qualities that are most important for your project.  You might have
  something like: ``build confidence in the software correctness,''
  ``demonstrate adequate usability.'' etc.  You won't list all of the qualities,
  just those that are most important.}
  
The objectives of the Verification and Validation procedures are to,

\begin{itemize}
\item Establish confidence in software correctness.
\item Ensuring that the software meets the expected quality standards.
\end{itemize}

\subsection{Relevant Documentation}

\wss{Reference relevant documentation.  This will definitely include your SRS
  and your other project documents (MG, MIS, etc.)}

The requirements for the PID Controller software are captured in the Software
Requirements Specification \cite{SRS}

The software design information are captured in the Module Guide  \cite{MG}
and Module Interface Specification \cite{MIS} documents respectively.

\section{Plan}
	
\subsection{Verification and Validation Team}

\wss{You, your classmates and the course instructor.  Maybe your supervisor.}

The members of the VnV team for this project are listed in Table \ref{tab:VnVTeam}.

\begin{table}[]
\begin{tabular}{cc}
\hline
\multicolumn{1}{c|}{Team   Member} & Role \\ \hline
Naveen Ganesh Muralidharan      & Author    \\
Dr. Spencer Smith                     & Course Instructor and Domain Expert    \\
Ting-Yu Wu                               & Domain Expert    \\
Siddharth (Sid) Shinde               & Secondary reviewer - SRS    \\
Gabriela Sánchez Díaz	               & Secondary reviewer - VnV Plan \\   
\end{tabular}
\caption{Verification and Validation Team}
\label{tab:VnVTeam}
\end{table}

\subsection{SRS Verification Plan}

\wss{List any approaches you intend to use for SRS verification.  This may just
  be ad hoc feedback from reviewers, like your classmates, or you may have
  something more rigorous/systematic in mind..}
  
The SRS will be independently peer-reviewed by members of the VnV team, specifically,
Dr. Spencer Smith, Ting-Yu Wu, and Siddharth (Sid) Shinde.
 
Any issues identified during the review are tracked and verified in Github \cite{Github}.
 
\subsection{Design Verification Plan}

\wss{Plans for design verification}

The software for this project is auto-generated by the Drasil Software \cite{Drasil}. Therefore
manual verification of the design is not required.

\subsection{Implementation Verification Plan}

\wss{You should at least point to the tests listed in this document and the unit
  testing plan.}
  
The implemented software is tested as follows,

\begin{itemize}
\item Automated systems testing, where the corresponding test cases
are listed in section \ref{sec_systems_tests}. 
\item Automated unit testing, where the corresponding test cases are
listed in section \ref{sec_unit_tests}. 
\item Statement coverage check. 
\item Static code analysis.
\end{itemize}

\wss{In this section you would also give any details of any plans for static verification of
  the implementation.  Potential techniques include code walk throughs, code
  inspection, static analyzers, etc.}

\subsection{Automated Testing and Verification Tools} \label{sec_Tools}

\wss{What tools are you using for automated testing.  Likely a unit testing
  framework and maybe a profiling tool, like ValGrind.  Other possible tools
  include a static analyzer, make, continuous integration tools, test coverage
  tools, etc.  Explain your plans for summarizing code coverage metrics.
  Linters are another important class of tools.  For the programming language
  you select, you should look at the available linters.  There may also be tools
  that verify that coding standards have been respected, like flake9 for
  Python.}
\wss{The details of this section will likely evolve as you get closer to the
  implementation.}

The tools utilized for verification are listed below,

\begin{itemize}
\item Systems Testing - Pytest \cite{Pytest} will be used for automated systems
testing. Since this  is a blackbox test, Pytest will be used at the stdin, stdout, and
stderr levels.

\item Unit Testing - Pytest \cite{Pytest} will be used for automated unit testing.
 Since this is a whitebox test, Pytest will be used at the API level.

\item Statement coverage - PyTest-Cov \cite{PyTest-Cov}. PyTest-Cov is used 
along with Pytest to obtain the statement coverage.

\item Memory leaks - Valgrind \cite{Valgrind} can be used for memory leak analyses. 

\item Linting - Flake8 \cite{Flake8}. Linting tool that checks for the coding style against
the PEP8 standard.

\end{itemize}

\subsection{Software Validation Plan}

\wss{If there is any external data that can be used for validation, you should
  point to it here.  If there are no plans for validation, you should state that
  here.}

There are no plans for the Validation of the \progname{} software.

\section{System Test Description} \label{sec_systems_tests}
	
\subsection{Tests for Functional Requirements}

\wss{Subsets of the tests may be in related, so this section is divided into
  different areas.  If there are no identifiable subsets for the tests, this
  level of document structure can be removed.}

This section contains the systems test cases for the functional requirements in the
SRS \cite{SRS}. The test cases are organized into two categories, the input-output 
tests and simulation parameters test.

\wss{Include a blurb here to explain why the subsections below
  cover the requirements.  References to the SRS would be good.}

\subsubsection{Input Output tests}

\wss{It would be nice to have a blurb here to explain why the subsections below
  cover the requirements.  References to the SRS would be good.  If a section
  covers tests for input constraints, you should reference the data constraints
  table in the SRS.}
  
This section verifies section 4.2.6, and requirements R1 and R2 of section 5.1
in the SRS \cite {SRS}. Various inputs are provided to the Software Under Test, 
and the output is verified. 
		
\paragraph{Input Constraints test}

\begin{table}[]
\begin{tabular}{|c|c|c|c|c|c|c|c|}
\hline
\multicolumn{1}{|l|}{} & \multicolumn{5}{c|}{Inputs}                    & \multicolumn{2}{c|}{Outputs} \\ \hline
ID                     & r\_t    & K\_p    & K\_d    & t\_sim & t\_step & y\_t      & Error\_Message   \\ \hline
TC-PID-1-1             & 1       & 20      & 1       & 10     & 0.01    & 0.9524    & N/A              \\ \hline
TC-PID-1-2             & -0.0001 & 20      & 1       & 10     & 0.01    & N/A       & InputError       \\ \hline
TC-PID-1-3             & 1       & -0.0001 & 1       & 10     & 0.01    & N/A       & InputError       \\ \hline
TC-PID-1-4             & 1       & 20      & -0.0001 & 10     & 0.01    & N/A       & InputError       \\ \hline
TC-PID-1-5             & 1       & 20      & 1       & 61     & 0.01    & N/A       & InputError       \\ \hline
TC-PID-1-6             & 1       & 20      & 1       & 0.5    & 0.01    & N/A       & InputError       \\ \hline
TC-PID-1-7             & 1       & 20      & 1       & 10     & 0.001   & N/A       & InputError       \\ \hline
TC-PID-1-8            & 1       & 20      & 1       & 10     & 2       & N/A       & InputError       \\ \hline
\end{tabular}
\end{table}

\begin{itemize}
\item{TC-PID-1\\}
\begin{itemize}

\item{\textbf{Control:}} Automatic
					
\item{\textbf{Initial State:}} None
					
\item{\textbf{Input:}} Set all inputs of the software to the values specified in the 
`Input' columns in Table-\ref{tbltc001}.
					
\item{\textbf{Output:}}  Verify that all the outputs of the software match the 
values specified in the `Output' columns of Table-\ref{tbltc001}.

\wss{The expected result for the given inputs}

\item{\textbf{Requirement ID(s):}}  R1, R2, R3, R4, R5.

\item{\textbf{Test Case Derivation:}}  This test case is to test the behavior of the system 
when the system is supplied with inputs that are out of range. The system should either
produce an output or an error message. 

For test cases with output, a relative error of 10\% is applied to accommodate rounding off errors,
and floating point representation errors.

\wss{Justify the expected value given in the Output field}
					
\item{\textbf{How test will be performed:}}  The test will be automated with Pytest as mentioned in 
section \ref{sec_Tools}. 
					
\end{itemize}
\end{itemize}

\paragraph{Output test}

\begin{itemize}
\item{TC-PID-2\\}
\begin{itemize}

\item{\textbf{Control:}} Automatic
					
\item{\textbf{Initial State:}} None

\item{\textbf{Requirement ID(s):}}  R1, R2, R3, R4, R5.
					
\item{\textbf{Input:}} Set the inputs to the software as follows,

$SP$ = 5
~\newline$K_p$ = 1
~\newline$K_i$ = 2
~\newline$K_d$ = 0
~\newline$T_\text{sim}$ = 30 s
~\newline$t_\text{step}$ = 0.01 s
					
\item{\textbf{Output:}}  Verify that the value in the last index of the following lists are,

$y(t)$ =  5 +/- 0.01
~\newline$T_\text{elapsed}$  = 30 +/- 0.01 s

\item{\textbf{Test Case Derivation:}}  The inputs of this test case have been calibrated 
with a free PID simulation model \cite{Free_PID}. Therefore in this test case, at the end 
of the simulation time, the Process Variable ($y(t)$) will be equal to the Set-Point. 

A relative error of 10\% is applied to the process variable to accommodate rounding off errors 
and floating point representation errors. Similarly, the simulation should end in the specified time, 
with a relative error of 10\% to accommodate rounding off errors, floating point representation errors, and
 operating system timer resolution.

\wss{Justify the expected value given in the Output field}

\item{\textbf{How test will be performed:}}  The test will be automated with Pytest as mentioned in 
section \ref{sec_Tools}.

\end{itemize}
\end{itemize}

\subsubsection{Simulation parameters test}
This section tests the simulation parameters, namely, the step time and the simulation time.

\paragraph{Simulation step time test}

\begin{itemize}
\item{TC-PID-4\\}
\begin{itemize}

\item{\textbf{Control:}} Automatic
					
\item{\textbf{Initial State:}} None

\item{\textbf{Requirement ID(s):}}  R1, R2, R3, R4, R5.
					
\item{\textbf{Input:}} Set the input to the software as follows,

$SP$ = 5
~\newline$K_p$ = 1
~\newline$K_i$ = 1
~\newline$K_d$ = 0
~\newline$T_\text{sim}$ = 10 s
~\newline$t_\text{step}$ = 0.5 s
					
\item{\textbf{Output:}}  Verify that the difference between the subsequent values of the 
$T_\text{elapsed}$ is 0.5 s.

\item{\textbf{Test Case Derivation:}}  $t_\text{step}$ is the time for one iteration of the of the
control loop; all computations are $t_\text{step}$ apart.

A relative error of 10\% is applied to the process variable for loss
of precision for rounding off errors and floating point representation
errors. Similarly, the simulation should end in the specified time, with a 
relative error of 10\% for rounding off errors, floating point 
representation errors, and operating system timer resolution.

\wss{Justify the expected value given in the Output field}

\item{\textbf{How test will be performed:}}  The test will be automated using Pytest as specified in 
section \ref{sec_Tools}.

\end{itemize}
\end{itemize}
...

\subsection{Tests for Nonfunctional Requirements}

\wss{The nonfunctional requirements for accuracy will likely just reference the
  appropriate functional tests from above.  The test cases should mention
  reporting the relative error for these tests.}

This section contains the test cases for the non-functional requirements (section
5.2) of the SRS \cite{SRS}.

\wss{Tests related to usability could include conducting a usability test and
  survey.}

\subsubsection{Software Quality Tests}
		
\paragraph{Statement Coverage Test}

\begin{itemize}
\item{TC-PID-5\\}
\begin{itemize}

\item{\textbf{Type:}} Dynamic, Automated.
					
\item{\textbf{Initial State:}} None

\item{\textbf{Requirement ID(s):}} NFR-6
					
\item{\textbf{Input/Condition:}}  Execute test cases TC-PID-1 to TC-PID-5.
					
\item{\textbf{Output/Result:}} Verify that the statement coverage is 100\%.

\item{\textbf{How test will be performed:}}  This test will be executed with Pytest \cite{Pytest} in 
the Pycharm \cite{Pycharm} IDE. The tests should be executed with the code coverage 
option. 

\end{itemize}
\end{itemize}

\paragraph{Static code analysis}

\begin{itemize}
\item{TC-PID-6\\}
\begin{itemize}

\item{\textbf{Type:}} Static, Automated.
					
\item{\textbf{Initial State:}} None

\item{\textbf{Requirement ID(s):}} NFR-6
					
\item{\textbf{Input/Condition:}}  Execute TC-PID-2
					
\item{\textbf{Output/Result:}} Verify that the inspection report does not contain any warnings.

\item{\textbf{How test will be performed:}}  This test will be executed in PyCharm \cite{Pycharm}. 
The default profile for verification should be selected. 

\end{itemize}
\end{itemize}

\subsubsection{Portability Test}

\begin{itemize}
\item{TC-PID-7\\}
\begin{itemize}

\item{\textbf{Type:}} Manual, Dynamic.
					
\item{\textbf{Initial State:}} None

\item{\textbf{Requirement ID(s):}} NFR-1
					
\item{\textbf{Input/Condition:}}  Execute TC-PID-2 in each of the the following 
operating systems,

\begin{enumerate}
\item Windows 10.
\item Bodhi Linux 5.1.0. 
\end {enumerate}
					
\item{\textbf{Output/Result:}} Verify that on each of the operating systems the test 
case TC-PID-2 passes. 

\item{\textbf{How test will be performed:}}  Manual execution of the test script on 
each of the operating systems.

\end{itemize}
\end{itemize}

\subsubsection{Modularity Test}
\begin{itemize}
\item{TC-PID-8\\}
\begin{itemize}
\item{\textbf{Type:}} Manual.
					
\item{\textbf{Initial State:}} None

\item{\textbf{Requirement ID(s):}} NFR-2
					
\item{\textbf{Input/Condition:}}  Execute TC-PID-2.
					
\item{\textbf{Output/Result:}} Verify that the test case TC-PID-2 passes.

\item{\textbf{How test will be performed:}}  
\begin{enumerate}
\item Create a new python script. This script may be one of the 
systems tests listed in section \ref{sec_systems_tests}. 
\item Import the PID controller software as a module in the created script.
\item Feed the same inputs as specified in TC-PID-2.
\end{enumerate}
\end{itemize}
\end{itemize}

\subsubsection{Security Tests}

\paragraph{Memory leak check}
\begin{itemize}
\item{TC-PID-9\\}
\begin{itemize}
\item{\textbf{Type:}} Automated, Dynamic.
					
\item{\textbf{Initial State:}} None

\item{\textbf{Requirement ID(s):}} NFR-3
					
\item{\textbf{Input/Condition:}}  Execute the test script of TC-PID-2.
					
\item{\textbf{Output/Result:}} Verify that the test case TC-PID-2 passes and there
are no memory leaks identified in the  report generated by Valgrind.

\item{\textbf{How test will be performed:}} Valgrind \cite{Valgrind} is used to fork
the python script. After the execution, a report is generated by Valgrind. 
\end{itemize}
\end{itemize}

\paragraph{Divide by-zero check}
\begin{itemize}
\item{TC-PID-10\\}
\begin{itemize}
\item{\textbf{Type:}} Static, Inspection.
					
\item{\textbf{Initial State:}} None

\item{\textbf{Requirement ID(s):}} NFR-3
					
\item{\textbf{Input/Condition:}}  Source code of the PID controller.
					
\item{\textbf{Output/Result:}} For every division in the source code, verify that
the denominator is tested for a non zero value. 

\item{\textbf{How test will be performed:}}  Manual inspection of the source code.
\end{itemize}
\end{itemize}

\paragraph{Negative square root check}
\begin{itemize}
\item{TC-PID-11\\}
\begin{itemize}
\item{\textbf{Type:}} Static analysis and Inspection
					
\item{\textbf{Initial State:}} None

\item{\textbf{Requirement ID(s):}} NFR-3
					
\item{\textbf{Input/Condition:}}  Source code of the PID controller.
					
\item{\textbf{Output/Result:}} For every square root function in the source code,
verify that the operands are tested for negative values before the function call.

\item{\textbf{How test will be performed:}}  Manual inspection of the source code.
\end{itemize}
\end{itemize}
\subsubsection{Maintainability test}

\begin{itemize}
\item{TC-PID-11\\}
\begin{itemize}
\item{\textbf{Type:}} Inspection
					
\item{\textbf{Initial State:}} None

\item{\textbf{Requirement ID(s):}} NFR-4
					
\item{\textbf{Input/Condition:}}  The source code, and the User's Guide of the PID controller.
					
\item{\textbf{Output/Result:}} Verify that all the classes, methods and modules in the source
code are documented in the User's Guide.

\item{\textbf{How test will be performed:}}  Manual inspection of the source code and the 
User's guide.
\end{itemize}
\end{itemize}

\subsubsection{Verifiability test}

\begin{itemize}
\item{TC-PID-13\\}
\begin{itemize}
\item{\textbf{Type:}} Inspection
					
\item{\textbf{Initial State:}} None

\item{\textbf{Requirement ID(s):}} NFR-5
					
\item{\textbf{Input/Condition:}} VnV report and the SRS of the PID Controller.
					
\item{\textbf{Output/Result:}} Verify that for each requirement in the SRS, there
exists at-least one test case in the VnV report.
 
\item{\textbf{How test will be performed:}} Manual inspection of the documents.
\end{itemize}
\end{itemize}
\subsection{Traceability Between Test Cases and Requirements}

\wss{Provide a table that shows which test cases are supporting which
  requirements.}

Table-\ref{tblTrace} contains the mapping of requirements to test cases.
  
\begin{table}[]
\caption{Requirements vs Test Cases Trace Matrix}
\label{tblTrace}
\begin{tabular}{|c|c|c|c|c|c|c|c|c|c|c|c|}
\hline
          & R1 & R2 & R3 & R4 & R5 & NFR1 & NFR2 & NFR3 & NFR4 & NFR5 & NFR6 \\ \hline
TC-PID-1  & X  & X  & X  & X  & X  &      &      &      &      &      &      \\ \hline
TC-PID-2  & X  & X  & X  & X  & X  &      &      &      &      &      &      \\ \hline
TC-PID-4  & X  & X  & X  & X  & X  &      &      &      &      &      &      \\ \hline
TC-PID-5  & X  & X  & X  & X  & X  &      &      &      &      &      &      \\ \hline
TC-PID-6  &    &    &    &    &    &      &      &      &      &      & X    \\ \hline
TC-PID-7  &    &    &    &    &    & X    &      &      &      &      &      \\ \hline
TC-PID-8  &    &    &    &    &    &      & X    &      &      &      &      \\ \hline
TC-PID-9  &    &    &    &    &    &      &      & X    &      &      &      \\ \hline
TC-PID-10 &    &    &    &    &    &      &      & X    &      &      &      \\ \hline
TC-PID-11 &    &    &    &    &    &      &      & X    &      &      &      \\ \hline
TC-PID-12 &    &    &    &    &    &      &      &      & X    &      &      \\ \hline
TC-PID-13 &    &    &    &    &    &      &      &      &      & X    &      \\ \hline
\end{tabular}
\end{table}

\section{Unit Test Description}  \label{sec_unit_tests}

\wss{Reference your MIS and explain your overall philosophy for test case
  selection.}  
\wss{This section should not be filled in until after the MIS has
  been completed.}

\subsection{Unit Testing Scope}

\wss{What modules are outside of the scope.  If there are modules that are
  developed by someone else, then you would say here if you aren't planning on
  verifying them.  There may also be modules that are part of your software, but
  have a lower priority for verification than others.  If this is the case,
  explain your rationale for the ranking of module importance.}

\subsection{Tests for Functional Requirements}

\wss{Most of the verification will be through automated unit testing.  If
  appropriate specific modules can be verified by a non-testing based
  technique.  That can also be documented in this section.}

\subsubsection{Module 1}

\wss{Include a blurb here to explain why the subsections below cover the module.
  References to the MIS would be good.  You will want tests from a black box
  perspective and from a white box perspective.  Explain to the reader how the
  tests were selected.}

\begin{enumerate}

\item{test-id1\\}

Type: \wss{Functional, Dynamic, Manual, Automatic, Static etc. Most will
  be automatic}

\begin{enumerate}
\item{\textbf{Initial State:}} 
					
\item{\textbf{Input:}} 
					
\item{\textbf{Output:}}  \wss{The expected result for the given inputs}

\item{\textbf{Test Case Derivation:}}  \wss{Justify the expected value given in the Output field}

\item{\textbf{How test will be performed:}}  

\end{enumerate}
					
\item{test-id2\\}

Type: \wss{Functional, Dynamic, Manual, Automatic, Static etc. Most will
  be automatic}
					
\item{\textbf{Initial State:}} 
					
\item{\textbf{Input:}} 
					
\item{\textbf{Output:}}  \wss{The expected result for the given inputs}

\item{\textbf{Test Case Derivation:}}  \wss{Justify the expected value given in the Output field}

\item{\textbf{How test will be performed:}}  

\item{...\\}
    
\end{enumerate}

\subsubsection{Module 2}

...

\subsection{Tests for Nonfunctional Requirements}

\wss{If there is a module that needs to be independently assessed for
  performance, those test cases can go here.  In some projects, planning for
  nonfunctional tests of units will not be that relevant.}

\wss{These tests may involve collecting performance data from previously
  mentioned functional tests.}

\subsubsection{Module ?}
		
\begin{enumerate}

\item{test-id1\\}

Type: \wss{Functional, Dynamic, Manual, Automatic, Static etc. Most will
  be automatic}

\begin{itemize}	
\item{\textbf{Initial State:}} 
					
\item{\textbf{Input/Condition:}}  
					
Output/Result: 
					
\item{\textbf{How test will be performed:}}  

\end{itemize}
\end{enumerate}
					
\begin{enumerate}
\item{test-id2\\}

Type: Functional, Dynamic, Manual, Static etc.

\begin{itemize}
					
\item{\textbf{Initial State:}} 
					
\item{\textbf{Input:}} 
					
\item{\textbf{Output:}}  
					
\item{\textbf{How test will be performed:}}  

\end{itemize}
\end{enumerate}

\subsubsection{Module ?}

...

\subsection{Traceability Between Test Cases and Modules}

\wss{Provide evidence that all of the modules have been considered.}
				

\bibliographystyle{abbrvnat}
\bibliography {../../refs/References}

%\newpage

%\section{Appendix}

%This is where you can place additional information.

%\subsection{Symbolic Parameters}

%The definition of the test cases will call for SYMBOLIC\_CONSTANTS.
%Their values are defined in this section for easy maintenance.

%\subsection{Usability Survey Questions?}

%\wss{This is a section that would be appropriate for some projects.}

\end{document}
