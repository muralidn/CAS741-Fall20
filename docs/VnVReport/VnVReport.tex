\documentclass[12pt, titlepage]{article}



\usepackage{booktabs}
\usepackage{tabularx}
\usepackage{hyperref}
\hypersetup{
    colorlinks,
    citecolor=black,
    filecolor=black,
    linkcolor=red,
    urlcolor=blue
}
\usepackage[round]{natbib}

\usepackage{amsmath, mathtools}
\usepackage{amsfonts}
\usepackage{amssymb}
\usepackage{graphicx}
\usepackage{colortbl}
\usepackage{xr}
\usepackage{hyperref}
\usepackage{longtable}
\usepackage{xfrac}
\usepackage{float}
\usepackage{siunitx}
\usepackage{pdflscape}
\usepackage{afterpage}

\usepackage{longtable}
\usepackage{caption}
\usepackage{enumitem}
\usepackage{adjustbox}

\input{../Comments}
%% Common Parts

\newcommand{\progname}{PID Controller} % PUT YOUR PROGRAM NAME HERE %Every program
                                % should have a name


\begin{document}

\title{Test Report: \progname{}} 
\author{Naveen Ganesh Muralidharan}
\date{\today}
	
\maketitle

\pagenumbering{roman}

\section{Revision History}

\begin{tabularx}{\textwidth}{p{3cm}p{2cm}X}
\toprule {\bf Date} & {\bf Version} & {\bf Notes}\\
\midrule
14-Dec-2020 & 1.0 & First Draft of the VnV Report\\

\bottomrule
\end{tabularx}

~\newpage

\section{Symbols, Abbreviations and Acronyms}

\renewcommand{\arraystretch}{1.2}
\begin{tabular}{l l} 
  \toprule		
  \textbf{symbol} & \textbf{description}\\
  \midrule 
  T & Test\\
  \bottomrule
\end{tabular}\\

\wss{symbols, abbreviations or acronyms -- you can reference the SRS tables if needed}
All the units, abbreviations, and symbols recorded in the Software Requirement 
Specification \cite{SRS} and the Verification and Validation Plan \cite{VnVPlan}  
apply to this document as well.


\newpage

\tableofcontents

\listoftables %if appropriate

\newpage

\pagenumbering{arabic}

This document provides the summary of the Verification and Validation of the 
\progname{} software. 

Sections 3 and 4 summarize the results of the functional and non-functional
requirements respectively. Subsequent sections summarizes the test results
in detail.

\section{Functional Requirements Evaluation}

In the VnV Plan  \cite{VnVPlan}  the test cases TC-PD-1, TC-PD-2, and TC-PD-14
test the functional requirements. The test results and the corresponding 
artifacts can be found here,

\url{https://github.com/muralidn/CAS741-Fall20/tree/master/test/results/functional_requirements}

\section{Nonfunctional Requirements Evaluation}

\subsection{Portability}

The \progname{} software was successfully executed on both Windows and Linux 
Operating Systems. The test results and the corresponding 
artifacts can be found here,

 \url{https://github.com/muralidn/CAS741-Fall20/tree/master/test/results/nonfunctional_requirements/portability}
		
\subsection{Maintenance}

\paragraph{Modularity} - 

The \progname{} software, generated by the Drasil \cite{Drasil} software, was
inspected, and it was found that the resultant code is modularized as follows,

\begin{itemize}

\item Calculations.py  - Provides functions for calculating the outputs.

\item Constants.py  - Provides the structure for holding constant values.

\item Control.py  - Controls the flow of the program.

\item InputParameters.py - Provides the function for reading inputs and the function 
for checking the physical constraints on the input.

\item OutputFormat.py  -  Provides the function for writing outputs.

\end{itemize}

\paragraph{Linting} - Linting check of the  \progname{} software against 
the PEP-8  standards did not reveal any significant issues. Minor issues such 
a white-space before `:', or long line lengths were identified. These are 
negligible, and hence overall the Linting test was successful. The test results 
and the corresponding artifacts can be found here,

 \url{https://github.com/muralidn/CAS741-Fall20/tree/master/test/results/nonfunctional_requirements/linting}

\paragraph{Documented} - The source code of the \progname{} software has been 
adequately captured in the Doxygen pdf file. There are few minor issues like missing
 names for a few functions but they are negligible.

The test results and the corresponding artifacts can be found here,

 \url{https://github.com/muralidn/CAS741-Fall20/tree/master/test/results/nonfunctional_requirements/linting}

\subsection{Security}

\paragraph{Memory Leak Check} - Valgrind analysis of the \progname{} software indicated
that there is about 576 Bytes confirmed leak, and 148 MB possible leak. However this 
data is inconclusive, as the leak maybe in Python itself.

The test results and the corresponding artifacts can be found here,

 \url{https://github.com/muralidn/CAS741-Fall20/tree/master/test/results/nonfunctional_requirements/memory_leak}

\paragraph{Negative square root check} - The source code of the \progname{} software was 
inspected, and there are no square root calls in the program.

\paragraph{Divide by Zero error} - The source code of the \progname{} software was 
inspected, and it was confirmed that the divide by zero error is averted by the checks for the 
input constraints.

\subsection{Verifiability}

\paragraph{Tracing} - All the functional and non functional requirements have at-least 
one test case in the VnV plan and report.

\paragraph{Statement Coverage} - Automated testing proved that 100\% statement
coverage has been achieved,

The artifacts of the Statement Coverage analysis are found here,

 \url{https://github.com/muralidn/CAS741-Fall20/tree/master/test/results/nonfunctional_requirements/code_coverage}

\paragraph{Data Coupling} - Inspection of the source code revealed that the functions listed in Table-\ref{tab:dataCoupling} 
are coupled by data flow,

\begin{table}[]
\caption{}
\label{tab:dataCoupling}
\begin{tabular}{c|c|c|c}
\multicolumn{1}{c|}{Module1} & \multicolumn{1}{c|}{Function1} & Module2         & Function2     \\ \hline
Control                      & main                           & InputParameters & get\_input    \\
Control                      & main                           & Calculations    & func\_y\_t    \\
Control                      & main                           & OutputFormatter & write\_output
\end{tabular}
\end{table}


The log file generated by the source code was inspected, and the DataFlow between the 
functions was confirmed.

\paragraph{Control Coupling} - Inspection of the source code revealed that the functions listed in Table-\ref{tab:ctrlCoupling}  
are coupled by control flow,

\begin{table}[]
\caption{}
\label{tab:ctrlCoupling}
\begin{tabular}{c|c|c|c}
\multicolumn{1}{c|}{Module1} & \multicolumn{1}{c|}{Function1} & Module2         & Function2     \\ \hline
Control                      & main                           & InputParameters & input\_constraints    \\
\end{tabular}
\end{table}

The log file generated by the source code was inspected, and the control flow in the 
function was confirmed.

The artifacts of the DCCC analysis are found here,

 \url{https://github.com/muralidn/CAS741-Fall20/tree/master/test/results/nonfunctional_requirements/DCCC}
	
\section{Comparison to Existing Implementation}	

Not applicable to the \progname{} software.

\section{Unit Testing}

The Unit test case, TC-PD-14, passed successfully. The results are found here,

  \url{https://github.com/muralidn/CAS741-Fall20/tree/master/test/results/functional_requirements}

\section{Changes Due to Testing}

The following bugs were caught during verification and were addressed,

\begin{itemize}

\item The maximum physical constraint of the  Step Time (${t_{\text{step}}}$) was 
updated to always be less than the Simulation Time (${t_{\text{sim}}}$). This way
the User is free to choose the step size.

\end {itemize}
		
\section{Trace to Requirements}

See section 5.3 of the VnV plan \cite{VnVPlan}.
		
\section{Trace to Modules}		

See section 6.3 of the VnV plan \cite{VnVPlan}.

\section{Code Coverage Metrics}

The statement coverage metric is summarized in Table-\ref{tab:statement-coverage}.

\begin{table}[]
\caption{}
\label{tab:statement-coverage}
\begin{tabular}{c|c}
Module             & Statement Coverage \\ \hline
Calculations.py    & 100\%              \\
Constants.py       & 100\%              \\
Control.py         & 100\%              \\
InputParameters.py & 100\%              \\
OutputFormat.py    & 100\%             
\end{tabular}
\end{table}

\bibliographystyle{plainnat}

\bibliography{../../refs/References}

\end{document}