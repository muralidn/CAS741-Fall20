\documentclass[12pt, titlepage]{article}

\usepackage{booktabs}
\usepackage{tabularx}
\usepackage{hyperref}


\usepackage{amsmath, mathtools}
\usepackage{amsfonts}
\usepackage{amssymb}
\usepackage{graphicx}
\usepackage{colortbl}
\usepackage{xr}
\usepackage{hyperref}
\usepackage{longtable}
\usepackage{xfrac}
\usepackage{float}
\usepackage{siunitx}
\usepackage{pdflscape}
\usepackage{afterpage}

\usepackage{longtable}
\usepackage{caption}
\usepackage{enumitem}
\usepackage{adjustbox}

\hypersetup{
    colorlinks,
    citecolor=black,
    filecolor=black,
    linkcolor=red,
    urlcolor=blue
}
\usepackage[square,numbers]{natbib}

%% Comments

\usepackage{color}

\newif\ifcomments\commentsfalse

\ifcomments
\newcommand{\authornote}[3]{\textcolor{#1}{[#3 ---#2]}}
\newcommand{\todo}[1]{\textcolor{red}{[TODO: #1]}}
\else
\newcommand{\authornote}[3]{}
\newcommand{\todo}[1]{}
\fi

\newcommand{\wss}[1]{\authornote{blue}{SS}{#1}} 
\newcommand{\plt}[1]{\authornote{magenta}{TPLT}{#1}} %For explanation of the template
\newcommand{\an}[1]{\authornote{cyan}{Author}{#1}}

%% Common Parts

\newcommand{\progname}{ProgName} % PUT YOUR PROGRAM NAME HERE %Every program
                                % should have a name


\begin{document}

\title{Test Report: \progname{}} 
\author{Naveen Ganesh Muralidharan}
\date{\today}
	
\maketitle

\pagenumbering{roman}

\section{Revision History}

\begin{tabularx}{\textwidth}{p{3cm}p{2cm}X}
\toprule {\bf Date} & {\bf Version} & {\bf Notes}\\
\midrule
15-Dec-2020 & 1.0 & First revision of the VnV Report\\

\bottomrule
\end{tabularx}

~\newpage

\section{Symbols, Abbreviations and Acronyms}

\renewcommand{\arraystretch}{1.2}
\begin{tabular}{l l} 
  \toprule		
  \textbf{symbol} & \textbf{description}\\
  \midrule 
  T & Test\\
 VnV & Verification and Validation\\
  \bottomrule
\end{tabular}\\

\wss{symbols, abbreviations or acronyms -- you can reference the SRS tables if needed}
All the units, abbreviations, and symbols recorded in the Software Requirement 
Specification \cite{SRS} and the Verification and Validation Plan \cite{VnVPlan}  
apply to this document as well.


\newpage

\tableofcontents

\listoftables %if appropriate

\newpage

\pagenumbering{arabic}

This document provides a summary of the Verification and Validation (VnV) of the 
\progname{} software. The test cases listed in the VnV plan \cite{VnVPlan} 
were executed, and this document contains a summary of the results.

Sections 3 and 4 summarize the results of the functional and non-functional
requirements respectively. Subsequent sections summarize the test results
in detail.

\section{Functional Requirements Evaluation}

\begin{itemize}

\item \textbf{Test case(s)}: TC-PD-1, TC-PD-2, and TC-PD-14.

\item \textbf{Requirements}: FR: Input-Values, FR: Verify-Input-Values,
FR: Calculate-Values, FR: Output-Values.

\item \textbf{Type}: Automated 

\item \textbf{Result Summary}: All the test cases passed successfully. 

\item \textbf{Results Artifacts Location}: \url{https://github.com/muralidn/CAS741-Fall20/tree/master/test/results/functional_requirements}

\end{itemize}

\section{Nonfunctional Requirements Evaluation}

\subsection{Portability}

\begin{itemize}

\item \textbf{Test case(s)}: TC-PD-3

\item \textbf{Requirements}: NFR: Portable

\item \textbf{Type}: Semi-automatic

\item \textbf{Result Summary}: The \progname{} software was successfully executed on both Windows and Linux 
Operating Systems.

 \item \textbf{Results Artifacts Location}: \url{https://github.com/muralidn/CAS741-Fall20/tree/master/test/results/nonfunctional_requirements/portability}
 
\end{itemize}
		
\subsection{Maintenance}

\paragraph{Modularity}

\paragraph{Data Coupling}

\begin{itemize}

\item \textbf{Test case(s)}: TC-PD-12

\item \textbf{Requirements}: NFR: Maintainable

\item \textbf{Type}: Manual/Inspection.

\item \textbf{Result Summary}: 

Inspection of the source code revealed that the functions listed in Table-\ref{tab:dataCoupling} 
are coupled by data flow.

\begin{table}[]
\caption{Data Coupling}
\label{tab:dataCoupling}
\begin{tabular}{c|c|c|c}
\multicolumn{1}{c|}{Module From} & \multicolumn{1}{c|}{Function From} & Module To         & Function To     \\ \hline
Control                      & main                           & InputParameters & get\_input    \\
Control                      & main                           & Calculations    & func\_y\_t    \\
Control                      & main                           & OutputFormatter & write\_output \\
Constants                   & Constants (class)         & Calculations    & func\_y\_t    \\
\end{tabular}
\end{table}

The log file generated by the source code was inspected, and the data flow between the 
functions was confirmed. 

\item \textbf{Results Artifacts Location}:  \url{https://github.com/muralidn/CAS741-Fall20/tree/master/test/results/nonfunctional_requirements/DCCC}

\end{itemize}

\paragraph{Control Coupling} 

\begin{itemize}

\item \textbf{Test case(s)}: TC-PD-13

\item \textbf{Requirements}: NFR: Maintainable

\item \textbf{Type}: Manual/Inspection.

\item \textbf{Result Summary}: 

Inspection of the source code revealed that the functions listed in Table-\ref{tab:ctrlCoupling}  
are coupled by control flow.

\begin{table}[]
\caption{Control Coupling}
\label{tab:ctrlCoupling}
\begin{tabular}{c|c|c|c}
\multicolumn{1}{c|}{Module From} & \multicolumn{1}{c|}{Function From} & Module To         & Function To     \\ \hline
Control                      & main                           & InputParameters & input\_constraints    \\
\end{tabular}
\end{table}

The log file generated by the source code was inspected, and the control flow in the 
function was confirmed.

\item \textbf{Results Artifacts Location}:  \url{https://github.com/muralidn/CAS741-Fall20/tree/master/test/results/nonfunctional_requirements/DCCC}

\end{itemize}

\paragraph{Linting}

\begin{itemize}

\item \textbf{Test case(s)}: TC-PD-5

\item \textbf{Requirements}: NFR: Maintainable

\item \textbf{Type}: Automated

\item \textbf{Result Summary}: 

 Linting check of the  \progname{} software against 
the PEP-8  standards did not reveal any significant issues. Minor issues such 
as whitespace before `:', or long line lengths were identified. These are 
negligible, and hence overall the Linting test was successful.

 \item \textbf{Results Artifacts Location}: \url{https://github.com/muralidn/CAS741-Fall20/tree/master/test/results/nonfunctional_requirements/linting}

\end{itemize}

\paragraph{Documented} 

\begin{itemize}

\item \textbf{Test case(s)}: TC-PD-6

\item \textbf{Requirements}: NFR: Maintainable

\item \textbf{Type}: Manual/Inspection

\item \textbf{Result Summary}: 

The source code of the \progname{} software has been 
adequately captured in the Doxygen pdf file. There are few minor issues like missing
 names for a few functions but they are negligible.

 \item \textbf{Results Artifacts Location}: \url{https://github.com/muralidn/CAS741-Fall20/tree/master/test/results/nonfunctional_requirements/linting}

\end{itemize}

\subsection{Security}

\paragraph{Memory Leak Check}

\begin{itemize}

\item \textbf{Test case(s)}: TC-PD-7

\item \textbf{Requirements}: NFR: Secure

\item \textbf{Type}: Automated

\item \textbf{Result Summary}: 

Valgrind analysis of the \progname{} software indicated
that there are about 576 Bytes confirmed leak, and 148 MB possible leak. However, this 
data is inconclusive, as the leak may be in Python itself.

\item \textbf{Results Artifacts Location}: \url{https://github.com/muralidn/CAS741-Fall20/tree/master/test/results/nonfunctional_requirements/memory_leak}

\end{itemize}

\paragraph{Negative square root check} 

\begin{itemize}

\item \textbf{Test case(s)}: TC-PD-9

\item \textbf{Requirements}: NFR: Secure

\item \textbf{Type}: Manual/Inspection

\item \textbf{Result Summary}: 

The source code of the \progname{} software was 
inspected, and there are no square root calls in the program.

\end{itemize}

\paragraph{Divide by Zero error} 

\begin{itemize}

\item \textbf{Test case(s)}: TC-PD-8

\item \textbf{Requirements}: NFR: Secure

\item \textbf{Type}: Manual/Inspection

\item \textbf{Result Summary}: 

The source code of the \progname{} software was 
inspected, and it was confirmed that the divide by zero error is averted by the checks for the 
input constraints.

\end{itemize}

\subsection{Verifiability}

\paragraph{Traceability}

\begin{itemize}

\item \textbf{Test case(s)}: TC-PD-10

\item \textbf{Requirements}: NFR: Verifiable

\item \textbf{Type}: Manual/Inspection

\item \textbf{Result Summary}: 

All the functional and non-functional requirements have at least 
one test case in the VnV plan \cite{VnVPlan} and report.

\end{itemize}

\paragraph{Statement Coverage}

\begin{itemize}

\item \textbf{Test case(s)}: TC-PD-11

\item \textbf{Requirements}: NFR: Verifiable

\item \textbf{Type}: Automated.

\item \textbf{Result Summary}: 

Automated testing proved that 100\% statement
coverage has been achieved,

\item \textbf{Results Artifacts Location}: \url{https://github.com/muralidn/CAS741-Fall20/tree/master/test/results/nonfunctional_requirements/code_coverage}

\end{itemize}



\section{Comparison to Existing Implementation}	

The results of the \progname{} software and an independent Simulink model (\cite{Simulink}, 
\cite{PD_Controller}) of the \progname{} are compared in test cases TC-PD-1, TC-PD-2 and 
TC-PD-14. The relative error between the outputs of the two software has been
verified to be less than 5\%.


\section{Unit Testing}

\begin{itemize}

\item \textbf{Test case(s)}: TC-PD-14

\item \textbf{Module}: Calculations.py

\item \textbf{Type}: Automated 

\item \textbf{Result Summary}: All the test cases passed successfully. 

\item \textbf{Results Artifacts Location}: \url{https://github.com/muralidn/CAS741-Fall20/tree/master/test/results/functional_requirements}

\end{itemize}

\section{Changes Due to Testing}

Bugs in the \progname{} software were continuously addressed throughout the course of the testing.
		
\section{Trace to Requirements}

Table-\ref{tblTrace} contains the mapping of requirements to test cases.
  
\begin{table}[]
\caption{Requirements vs Test Cases Trace Matrix}
\label{tblTrace}
\begin{adjustbox}{width=1.29\textwidth}
\begin{tabular}{c|c|c|c|c|c|c|c|c|}
 & Input-Values & Verify-Input-Values & Calculate-Values & Output-Values & Portable & Secure & Maintainable & Verifiable \\ \hline
TC-PD-1 & X & X & X  & X  &  &  &  &  \\
TC-PD-2 & X  & X  & X & X &  &  &  &  \\
TC-PD-3  &  &  &  &   & X  &  &  &  \\
TC-PD-5  &  &  &  &  &  &  &X  &  \\
TC-PD-6  &  &  &  &  &  &  &X  &  \\
TC-PD-7  &  &  &  &  &  &X  &  &  \\
TC-PD-8  &  &  &  &  &  &X  &  &  \\
TC-PD-9  &  &  &  &  &  &X  &  &  \\
TC-PD-10 &  &  &  &  &  &  &  &X  \\
TC-PD-11 &  &  &  &  &  &  &  &X \\
TC-PD-12 &  &  &   &   &  &  &  &X  \\
TC-PD-13 &   &  &  &  &  &  &  &X \\
TC-PD-14 &   &  &X  &  &  &  &  &X \\
\end{tabular}
\end{adjustbox}

\end{table}
		
\section{Trace to Modules}		

Table-\ref{tblTraceMod} contains the mapping of modules to test cases.
  
\begin{table}[]
\caption{Modules vs Test Cases Trace Matrix}
\label{tblTraceMod}
\begin{adjustbox}{width=1.29\textwidth}
\begin{tabular}{c|c|c|c|c|c|}
 & Calculations.py & Constants.py & Control.py & InputParameters.py & OutputFormat.py \\ \hline
TC-PD-1 & X & X & X  & X  & X\\
TC-PD-2 & X  & X  & X & X & X \\
TC-PD-3  &  &  &  &   & \\
TC-PD-5  &  &  &  &  &  \\
TC-PD-6  &  &  &  &  &   \\
TC-PD-7  &  &  &  &  &   \\
TC-PD-8  &  &  &  &  &   \\
TC-PD-9  &  &  &  &  &   \\
TC-PD-10 &  &  &  &  &  \\
TC-PD-11 &X  &X  &X  &X  &X  \\
TC-PD-12 &X  &X  &X   &X   &X  \\
TC-PD-13 &   &  &X  &X  &  \\
TC-PD-14 &X   &  &  &  &  \\
\end{tabular}
\end{adjustbox}
\end{table}

\section{Code Coverage Metrics}

The statement coverage metric is summarized in Table-\ref{tab:statement-coverage}.

\begin{table}[]
\caption{Statement Coverage Summary of \progname{}}
\label{tab:statement-coverage}
\begin{tabular}{cccc}
\hline
\multicolumn{1}{c|}{Module} & \multicolumn{1}{c|}{Statements} & \multicolumn{1}{c|}{Miss} & Coverage \\ \hline
Calculations.py    & 41  & 0 & 100\% \\
Constants.py       & 3   & 0 & 100\% \\
Control.py         & 19  & 0 & 100\% \\
InputParameters.py & 110 & 0 & 100\% \\
OutputFormat.py    & 11  & 0 & 100\% \\ \hline
TOTAL              & 184 & 0 & 100\%
\end{tabular}
\end{table}

\bibliographystyle{abbrvnat}
\bibliography{../../refs/References}

\end{document}